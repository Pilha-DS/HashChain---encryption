\documentclass[12pt,oneside,a4paper]{abntex2}

% ====================================================
% PACOTES E CONFIGURAÇÕES
% ====================================================

% --- Configurações básicas de idioma e codificação ---
\usepackage[brazil]{babel}
\usepackage[utf8]{inputenc}
\usepackage[T1]{fontenc}

% --- Pacotes essenciais ---
\usepackage{lmodern}
\usepackage{microtype}
\usepackage{graphicx}
\usepackage{float}
\usepackage{amsmath, amssymb}
\usepackage{geometry}
\usepackage{setspace}
\usepackage{indentfirst}
\usepackage{eso-pic}
\usepackage{fancyhdr}

% --- Pacote para texto placeholder (REMOVER NO FINAL) ---
\usepackage{lipsum}

% ====================================================
% CONFIGURAÇÕES DE FORMATAÇÃO
% ====================================================

% --- Configurações de geometria ---
\geometry{
	a4paper,
	left=3cm,
	right=2cm,
	top=3cm,
	bottom=2cm
}

% --- Configurações de espaçamento ---
\OnehalfSpacing
\setlength{\parindent}{1.25cm}

% --- SOLUÇÃO SIMPLES: Forçar numeração de capítulos ---
\makeatletter
% Garante que capítulos sejam numerados
\def\@chapapp{\chaptername}
\makeatother

% Configura a profundidade de numeração
\setcounter{secnumdepth}{3} % Até subseções

% --- Configurações ABNT personalizadas ---
\renewcommand{\fonte}[1]{%
	\begin{center}
		\footnotesize Autoria: #1
	\end{center}
}

% ====================================================
% METADADOS DO TRABALHO
% ====================================================

\titulo{Título do seu Projeto de Iniciação Científica}
\autor{Nome Completo}
\local{Cidade}
\data{Ano}

\instituicao{
	Universidade Paulista -- UNIP\\
	Curso de Ciência da Computação
}

\tipotrabalho{Trabalho de Iniciação Científica}
\orientador{Nome do Orientador}

\renewcommand{\resumoname}{\raggedright RESUMO}

% ====================================================
% DOCUMENTO
% ====================================================

\begin{document}
	
	% ====================================================
	% ELEMENTOS PRÉ-TEXTUAIS (sem numeração visível)
	% ====================================================
	
	\pagenumbering{roman}
	\pagestyle{empty}
	
	% --- Elementos pré-textuais ---
	\imprimircapa
	\imprimirfolhaderosto*
	
	\begin{resumo}
		Escreva aqui o resumo do seu projeto de IC.  
		Deve apresentar: objetivo, metodologia, resultados esperados e conclusão parcial.
		
		\textbf{Palavras-chave}: palavra 1; palavra 2; palavra 3.
	\end{resumo}
	
	\listoffigures*
	\clearpage
	
	\listoftables*
	\clearpage
	
	\tableofcontents*
	\clearpage
	
	% Remove QUALQUER linha de cabeçalho
	\renewcommand{\headrulewidth}{0pt}
	
	% Remove QUALQUER linha de rodapé
	\renewcommand{\footrulewidth}{0pt}
	
	% Estilo das páginas normais
	\pagestyle{fancy}
	\fancyhf{}
	\fancyfoot[R]{\thepage} % número no canto direito inferior
	
	% Estilo das páginas de capítulo (plain)
	\fancypagestyle{plain}{%
		\fancyhf{}
		\fancyfoot[R]{\thepage}
		\renewcommand{\headrulewidth}{0pt}
		\renewcommand{\footrulewidth}{0pt}
	}
	
	\pagenumbering{arabic}
	\setcounter{page}{6}
	
	% ====================================================
	% CORPO DO TEXTO (com numeração)
	% ====================================================
	
	% --- CAPÍTULO 1: INTRODUÇÃO ---
	\chapter{Introdução}
		\thispagestyle{plain}
		\label{cap:introducao}
		
		Este é o Capítulo 1 - Introdução. Deve mostrar "CAPÍTULO 1" no título.
		
		\lipsum[1-2]
		
		\begin{figure}[H]
			\centering
			\caption{Descrição da Figura}
			\includegraphics[width=0.8\textwidth]{example-image}
			\fonte{(nome da fonte)}
			\label{fig:exemplo1}
		\end{figure}
		
		\section{Primeira Seção}
			\label{sec:primeira}
			
			Esta é a Seção 1.1 - deve mostrar numeração.
			
			\lipsum[3-4]
			
			\subsection{Subseção}
				\label{subsec:primeira}
				
				Esta é a Subseção 1.1.1 - deve mostrar numeração.
				
				\lipsum[5-6]
				
		\section{segunda Seção}
			\label{sec:segunda}
			
			Esta é a Seção 1.2 - deve mostrar numeração.
			
			\lipsum[3-4]
			
			\subsection{Subseção segunda}
			\label{subsec:segunda}
			
			Esta é a Subseção 1.2.1 - deve mostrar numeração.
			
			\lipsum[5-6]
		
	% --- CAPÍTULO 2: DESENVOLVIMENTO ---
	\chapter{Desenvolvimento}
		\thispagestyle{plain}
		\label{cap:desenvolvimento}
		
		Este é o Capítulo 2 - Desenvolvimento. Deve mostrar "CAPÍTULO 2" no título.
		
		\lipsum[7-8]
		
		\begin{figure}[H]
			\centering
			\caption{Outra Figura}
			\includegraphics[width=0.8\textwidth]{example-image}
			\fonte{própria (2025)}
			\label{fig:exemplo2}
		\end{figure}
		
		\lipsum[9-10]
	
	% --- CAPÍTULO 3: CONCLUSÃO ---
	\chapter{Conclusão}
		\thispagestyle{plain}
		\label{cap:conclusao}
		
		Este é o Capítulo 3 - Conclusão. Deve mostrar "CAPÍTULO 3" no título.
		
		\lipsum[11-12]
	
\end{document}